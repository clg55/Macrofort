\documentstyle{article}

%\input{amssym.def}
%\input{amssym}

\textheight=23cm \textwidth=16cm
\topmargin=-1cm
\oddsidemargin=0pt \evensidemargin=0pt \marginparwidth=2cm

\def\four!{\!\!\!\!}
\def\eg{\hbox{e.g.}\ }
\def\is{\stackrel{\rm def}{=}}
\def\ie{\hbox{i.e.}\ }
\def\R{I\!\!R}

\title{MACROFORT : a FORTRAN code generator in MAPLE}

\author{Claude Gomez \\\\
        {\small INRIA} \\
        {\small Domaine de Voluceau, BP 105} \\
        {\small 78153 Le Chesnay Cedex} \\
        {\small France} \\
        {\small email: Claude.Gomez@inria.fr}}

\date{January 1992}

\begin{document}

\maketitle

\begin{abstract}
MACROFORT is a package which permits complete FORTRAN 77 code
generation in MAPLE. 
While using MACROFORT, the user stays in MAPLE. He no longer
deals with FORTRAN label numbering. Complicate FORTRAN loops are easy 
to do because 
{\tt while} and {\tt until} capabilities exist in MACROFORT. The generated
code can be optimized. Two examples are given. In the first one,
MAPLE is not able to solve numerically a problem using its own 
functions. In the second one, a naturally recursive problem is easily solved
by recursively generating a FORTRAN program.
\end{abstract}

%%%%%%%%%%%%%%%%%%%%%%%%%%%%%%%%%%%%%%%%%%%%%%%%%%%%%%%%%%%%%%%%%%%%%%%%%%%%%%
\section{Why generate FORTRAN ?}
%%%%%%%%%%%%%%%%%%%%%%%%%%%%%%%%%%%%%%%%%%%%%%%%%%%%%%%%%%%%%%%%%%%%%%%%%%%%%%
\label{whyf}

MAPLE is a computer algebra system and is well suited  to handle symbolic
computations. It is possible to perform numerical calculations in MAPLE
as well with floating-point numbers with a fixed number of digits (which 
can be as big as we want). A few MAPLE functions deal with numerical
computing, \eg {\tt evalf} and {\tt fsolve} which permit numerical 
integrations and numerical solutions of equations. But 
the running time of MAPLE in pure numerical computing is not as good as
the one of FORTRAN code. Moreover, if we want to write portable numerical
software and give it to another user who does not have MAPLE,
a good tractable language for industry is FORTRAN. 
Also, there are numerical problem that MAPLE fails to solve (see \ref{e1}).
Then, it would be nice to gather MAPLE and FORTRAN
in order to use the capabilities of MAPLE for computer algebra and the
capabilities of FORTRAN for numerical computing.

We can use MAPLE as a preprocessor for numerical analysis. 
For instance when we have to do a lot of symbolic manipulations as performing 
derivatives, integrals, before writing FORTRAN code.

There are problems which are naturally described in a recursive way.
If the problem is tail recursive, it is easy to solve it
in FORTRAN by using iterations. If it is not,
it is very complicated to do that (see \ref{e2}).

%%%%%%%%%%%%%%%%%%%%%%%%%%%%%%%%%%%%%%%%%%%%%%%%%%%%%%%%%%%%%%%%%%%%%%%%%%%%%%
\section{Why use MACROFORT ?}
%%%%%%%%%%%%%%%%%%%%%%%%%%%%%%%%%%%%%%%%%%%%%%%%%%%%%%%%%%%%%%%%%%%%%%%%%%%%%%
\label{whym}

There is a MAPLE procedure called {\tt fortran}.
It takes as argument a MAPLE expression and returns it in FORTRAN syntax.

For instance we have :
\begin{verbatim}
> e:=sin(tan(x^(y+1))-log(sin(x-1/y)))+Pi*x^(x^x)/(1.23-abs(y))+expand((1+x)^5);



                                                              x
                                                           (x )
                      (y + 1)                          Pi x
    e := - sin(- tan(x       ) + ln(sin(x - 1/y))) + ------------- + 1 + 5 x
                                                     1.23 - abs(y)

               2       3      4    5
         + 10 x  + 10 x  + 5 x  + x

> fortran([exp=e]);
      exp = -sin(-tan(x**(y+1))+alog(sin(x-1/y)))+Pi*x**x**x/(0.123E1-ab
     +s(y))+1+5*x+10*x**2+10*x**3+5*x**4+x**5
\end{verbatim}

So, we could use this function to generate the FORTRAN statements which do
the symbolic calculations into a file and then merge it with the remaining part
of the FORTRAN code. 
But we find this process not very easy because we have to use MAPLE,
to create intermediate files, to write FORTRAN code in another file and to
merge the files. It would be nice to stay in MAPLE for the whole process.
This can be done with MACROFORT.

%%%%%%%%%%%%%%%%%%%%%%%%%%%%%%%%%%%%%%%%%%%%%%%%%%%%%%%%%%%%%%%%%%%%%%%%%%%%%%
\section{MACROFORT}
%%%%%%%%%%%%%%%%%%%%%%%%%%%%%%%%%%%%%%%%%%%%%%%%%%%%%%%%%%%%%%%%%%%%%%%%%%%%%%
\label{macrofort}

Actually, we have made a kind of macro FORTRAN we have called MACROFORT. A
MAPLE user can perform symbolic calculations and write FORTRAN code without
leaving MAPLE. 
The original version of MACROFORT was made for the computer algebra system
MACSYMA (see \cite{macsyma}).

To generate FORTRAN code by using MACROFORT, we have to build a MAPLE list.
Each element of which corresponds to a single FORTRAN statement or to 
a few FORTRAN statement we call a macro FORTRAN statement.

The syntax of a FORTRAN statement or of a macro FORTRAN statement in 
MACROFORT is a MAPLE list where the first element is a keyword describing
the statement, the optional other elements are relevant arguments.
A keyword is made from the FORTRAN instruction name (when it exists)
with a {\tt f} at its end for a single FORTRAN statement or a {\tt m}
at its end for a macro FORTRAN statement. These keywords correspond to
what we call a MACROFORT single instruction or a MACROFORT macro instruction.

\subsection{MACROFORT single instructions}
%%%%%%%%%%%%%%%%%%%%%%%%%%%%%%%%%%%%%%

The MACROFORT single instructions are the following:

\begin{tabbing}
  {\tt [dof,{\rm label},{\rm index},{\rm start},
       {\rm end},{\rm step}]}\ \ \ \=
       generates\ \ \ \= \kill

  {\tt [callf,{\rm name},{\rm list}]} 
     \>generates
     \>{\tt\ \ \ \ \ \ call\ {\rm name}\ ({\rm list})}\\
  {\tt [closef,{\rm unit}]}
     \>generates
     \>{\tt\ \ \ \ \ \ close ({\rm unit})}\\
  {\tt [commentf,{\rm string}]} \>generates
    \>{\tt c\ \ \ \ {\rm string}}\\
  {\tt [commonf,{\rm name},{\rm list}]} 
     \>generates
     \>{\tt\ \ \ \ \ \ common\ /{\rm name}/\ {\rm list}}\\
  {\tt [continuef,{\rm label}]} \>generates
     \>{\tt{\rm label}\ continue}\\
  {\tt [declaref,{\rm type},{\rm list}]} 
     \>generates
     \>{\tt\ \ \ \ \ \ {\rm type}\ {\rm list}}\\
  {\tt [dof,{\rm label},{\rm index},{\rm start},{\rm end}]}
     \>generates
     \>{\tt\ \ \ \ \ \ do\ {\rm label}, 
       {\rm index}={\rm start}, {\rm end}}\\
  {\tt [dof,{\rm label},{\rm index},{\rm start},{\rm end},{\rm step}]}
     \>generates 
     \>{\tt\ \ \ \ \ \ do\ {\rm label}, 
       {\rm index}={\rm start}, {\rm end}, {\rm step}}\\
  {\tt [elsef]} \>generates\> {\tt\ \ \ \ \ \ else}\\
  {\tt [endf]} \>generates\> {\tt\ \ \ \ \ \ end}\\
  {\tt [endiff]} \>generates\> {\tt\ \ \ \ \ \ endif}\\
  {\tt [equalf,{\rm variable},{\rm expression}]} 
    \>generates\> 
    {\tt\ \ \ \ \ \ {\rm variable}={\rm expression}}\\
  {\tt [formatf,{\rm label},{\rm list}]} \>generates
    \>{\tt{\rm label}\ format\ ({\rm list})}\\
  {\tt [functionf,{\rm type},{\rm name},{\rm list}]} 
     \>generates
     \>{\tt\ \ \ \ \ \ {\rm type}\ function\ {\rm name}
       \ ({\rm list})}\\
  {\tt [gotof,{\rm label}]} \>generates 
    \>{\tt\ \ \ \ \ \ goto\ {\rm label}}\\
  {\tt [if\_goto\_f,{\rm condition},{\rm label}]} 
    \>generates
    \>{\tt\ \ \ \ \ \ if\ ({\rm condition})\ goto\ 
       {\rm label}}\\
  {\tt [if\_then\_f,{\rm condition}]} \>generates
    \>{\tt\ \ \ \ \ \ if\ ({\rm condition})\ then}\\
  {\tt [openf,{\rm unit},{\rm file},{\rm status}]}
     \>generates
     \>{\tt\ \ \ \ \ \ open\ (unit={\rm unit},
       file=\,'{\rm file}',}\\
     \> \> {\tt\ \ \ \ \ \ \ \ status=\,'{\rm status}')}\\
  {\tt [parameterf,{\rm list}]} 
     \>generates
     \>{\tt\ \ \ \ \ \ parameter\ ({\rm list})}\\
  {\tt [programf,{\rm name}]} \>generates
     \>{\tt\ \ \ \ \ \ program\ {\rm name}}\\
  {\tt [readf,{\rm file},{\rm label},{\rm list}]} 
    \>generates
    \>{\tt\ \ \ \ \ \ read\ ({\rm file},
      {\rm label})\ {\rm list}}\\
  {\tt [returnf]} \>generates\> {\tt\ \ \ \ \ \ return}\\
  {\tt [subroutinef,{\rm name},{\rm list}]} \>generates
     \>{\tt\ \ \ \ \ \ subroutine\ {\rm name}\ ({\rm list})}\\
  {\tt [writef,{\rm file},{\rm label},{\rm list}]} 
    \>generates
    \>{\tt\ \ \ \ \ \ write\ ({\rm file},
      {\rm label})\ {\rm list}}\\
\end{tabbing}


The arguments of MACROFORT instructions are MAPLE 
names and you have to quote them if
necessary.

When ``condition'' appears as an argument of a MACROFORT instruction,
you only have to write it with MAPLE syntax and MACROFORT will do
FORTRAN translation. For instance:

{\tt [if\_then\_f,a>=b]}.

When you want to introduce logical operators {\tt not}, {\tt and} and 
{\tt or} in a condition, you have to use the names {\tt NOT}, {\tt AND}
and {\tt OR} with a functional notation. 
For instance:

{\tt [if\_then\_f,OR(a=b,NOT(c<d))]}.

The label numbers are automatically generated by MACROFORT. When ``label''
appears as an argument of a MACROFORT instruction, you have to put a MAPLE
name. A same name corresponds to a same label and MACROFORT will generate
the label number. In fact we will see latter (see \ref{macro}) that we
can always avoid using labels.

When ``list'' appears as an argument of a MACROFORT instruction, it
corresponds to an argument FORTRAN list which you have to write as a
MAPLE list. For instance:

{\tt [callf,foo,[a,b,c]]}\\or

{\tt [formatf,[`2x,e14.7`],[x,y]]}.

When you want to generate a FORTRAN array, you have to write a MAPLE array.

All this is also available for MACROFORT macro instructions. But for them
no label is needed.
 
\subsection{MACROFORT macro instructions}
%%%%%%%%%%%%%%%%%%%%%%%%%%%%%%%%%%%%%%
\label{macro}

The MACROFORT macro instructions are the following:

\begin{tabbing}
  {\tt [if\_then\_else\_m,{\rm condition},{\rm then\_list},
       {\rm else\_list}]}\ \ \=generates\ \ \= \kill 

  {\tt [dom,{\rm index},{\rm start},{\rm end},
       {\rm step},{\rm do\_list}]}
     \>generates 
     \>{\tt\ \ \ \ \ \ do\ {\rm label}, 
       {\rm index}={\rm start}, {\rm end}, {\rm step}}\\
  \> \>{\tt\ \ \ \ \ \ \ }{\rm do\_list}\\
  \> \>{\tt{\rm label}\ continue}\\\\
  {\tt [dom,{\rm index},{\rm start},{\rm end},
       {\rm do\_list}]}
     \>generates 
     \>{\tt\ \ \ \ \ \ do\ {\rm label}, 
       {\rm index}={\rm start},{\rm end}}\\
    \> \>{\tt\ \ \ \ \ \ \ \ }{\rm do\_list}\\
    \> \>{\tt{\rm label}\ continue}\\\\
  {\tt [functionm,{\rm type},{\rm name},{\rm list},
       {\rm body\_list}]} 
     \>generates
     \>{\tt\ \ \ \ \ \ {\rm type}\ function\ {\rm name}
       \ ({\rm list})}\\
    \> \>{\tt\ \ \ \ \ \ \ }{\rm body\_list}\\
    \> \>{\tt\ \ \ \ \ \ end}\\\\
  {\tt [if\_then\_else\_m,{\rm condition},{\rm then\_list},
       {\rm else\_list}]}
     \>generates 
     \>{\tt\ \ \ \ \ \ if\ {\rm condition}\ then}\\
    \> \>{\tt\ \ \ \ \ \ \ \ }{\rm then\_list}\\
    \> \>{\tt\ \ \ \ \ \ else}\\
    \> \>{\tt\ \ \ \ \ \ \ \ }{\rm else\_list}\\
    \> \>{\tt\ \ \ \ \ \ endif}\\\\
  {\tt [if\_then\_m,{\rm condition},{\rm then\_list}]}
     \>generates 
     \>{\tt\ \ \ \ \ \ if\ {\rm condition}\ then}\\
    \> \>{\tt\ \ \ \ \ \ \ \ }{\rm then\_list}\\
    \> \>{\tt\ \ \ \ \ \ endif}\\\\
  {\tt [programm,{\rm name},{\rm body\_list}]} \>generates
     \>{\tt\ \ \ \ \ \ program\ {\rm name}}\\
    \> \>{\tt\ \ \ \ \ \ \ }{\rm body\_list}\\
    \> \>{\tt\ \ \ \ \ \ end}\\\\
  {\tt [openm,{\rm unit},{\rm file},{\rm status},
     {\rm body\_list}]} \>generates
     \>{\tt\ \ \ \ \ \ open\ (unit={\rm unit},
       file=\,'{\rm file}',}\\
    \> \>{\tt\ \ \ \ \ \ \ \ status=\,'{\rm status}')}\\
    \> \>{\tt\ \ \ \ \ \ \ }{\rm body\_list}\\
    \> \>{\tt\ \ \ \ \ \ close ({\rm unit})}\\\\
  {\tt [readm,{\rm file},{\rm format\_list},
       {\rm var\_list}]} 
     \>generates
     \>{\tt\ \ \ \ \ \ read\ ({\rm file},{\rm label})
       \ {\rm var\_list}}\\
    \> \>{\tt{\rm label}\ format\ ({\rm format\_list})}\\\\
  {\tt [subroutinem,{\rm name},{\rm list},{\rm body\_list}]} 
     \>generates
     \>{\tt\ \ \ \ \ \ subroutine\ {\rm name}\ ({\rm list})}\\
    \> \>{\tt\ \ \ \ \ \ \ }{\rm body\_list}\\
    \> \>{\tt\ \ \ \ \ \ end}\\\\
  {\tt [writem,{\rm file},{\rm format\_list},
       {\rm var\_list}]} 
     \>generates
     \>{\tt\ \ \ \ \ \ write\ ({\rm file},{\rm label})
       \ {\rm var\_list}}\\
    \> \>{\tt{\rm label}\ format\ ({\rm format\_list})}
\end{tabbing}

There are also the following macro instructions:
\begin{tabbing}
  {\tt [commonm,{\rm name},{\rm list}]}\ \ \=generates\ \ \= \kill 

  {\tt [commonm,{\rm name},{\rm list}]} 
     \>generates
     \>{\tt\ \ \ \ \ \ common\ /{\rm name}/\ {\rm list}}\\
  {\tt [declarem,{\rm type},{\rm list}]} 
     \>generates
     \>{\tt\ \ \ \ \ \ {\rm type}\ {\rm list}}
\end{tabbing}
The only difference with {\tt commonf} and {\tt declaref} single
instructions is that you can put these macros everywhere in the list
describing the program and MACROFORT put them at the right place
in the generated FORTRAN code. This permits to declare a variable
only when it is used in the body of the program. These macros only
work within a {\tt programm}, {\tt functionm} or {\tt subroutinem}
macro instruction, otherwise there are ignored.

There are other very important macro instructions. We do not give the
corresponding generated FORTRAN code because it is complicated.

First there are two macro instructions corresponding to WHILE and UNTIL
loops. We explain below their semantic with a PASCAL like syntax:

\smallskip
{\tt [whilem,{\rm condition},{\rm init\_list},{\rm while\_list},{\rm
while\_max}]} :

\quad $<$init\_list$>$\par
\quad {\bf while} condition {\bf do}\par
\qquad  $<$while\_list$>$\par
\quad {\bf end.}

\smallskip
{\tt [untilm,{\rm condition},{\rm init\_list},{\rm until\_list},{\rm 
until\_max}]} :

\quad $<$init\_list$>$\par
\quad {\bf do} $<$until\_list$>$\par
\qquad  {\bf until} condition\par
\quad {\bf end.}

``while\_max'' and ``until\_max'' are optional arguments and are the maximum 
number of iterations
that the WHILE loop or  the UNTIL loop will execute. When the maximum 
is reached,
the loop stops and a message is issued. If this argument is not given, there
is no maximum number of iterations.

``do\_list'', ``then\_list'', ``else\_list'', 
``body\_list'', 
``init\_list'', ``until\_list'' and ``while\_list'' arguments
must be MAPLE lists describing FORTRAN statements with MACROFORT syntax. 

You can nest as many loops as you want.

{\tt [matrixm,{\rm variable},{\rm matrix}]} is another very useful
macro instruction. It is used to make assignations of the elements
of a matrix. ``variable'' is the name of a FORTRAN matrix and ``matrix''
is the name of a MAPLE matrix.

For example, after:
\begin{verbatim}
a:=array([[x^2,x-y],[x/y,x^2-1]]);
                                   [  2          ]
                                   [ x    x - y  ]
                              a := [             ]
                                   [       2     ]
                                   [ x/y  x  - 1 ]

\end{verbatim}
{\tt [matrixm,v,a]} generates:
\begin{verbatim}
      v(2,2) = x**2-1
      v(1,2) = x-y
      v(2,1) = x/y
      v(1,1) = x**2
\end{verbatim}

An example of the use of {\tt whilem} and {\tt matrixm} is given
in section \ref{e1}.

In fact, a great number of 
single instructions (for instance {\tt dof}, {\tt if\_then\_f}, 
{\tt writef} \ldots)
are never used because macro instructions are
better and by using them it is never necessary to use labels.

\subsection{Flags and global variables}
%%%%%%%%%%%%%%%%%%%%%%%%%%%%%%%%%%%%%%%

There are global variables defined by MACROFORT.

\begin{description}
  \item[comment] is a logical variable. When it is  {\tt true} 
(default value) MACROFORT generates automatically FORTRAN comments.
  \item[input] is the logical unit number of standard input (the default
is 5).
  \item[output] is the logical unit number of standard output (the default
is 6).
  \item[optimized] is a logical variable. When it is {\tt true} (the default
value is {\tt false}) MAPLE optimizer ({\tt optimize} MAPLE function)
is used when FORTRAN is generated.
  \item[precision] specifies single or double precision in the generated
FORTRAN program. It can be {\tt single} (the default value) or {\tt double}.
\end{description}

When double precision is used with MAPLE optimizer, you must know that
MAPLE optimizer generates variables beginning with the letter
``t''. So you can add a {\tt implicit doubleprecision(t)} declaration 
in the FORTRAN code by using:
 
{\tt [declaref,`implicit doubleprecision`,[`(t)`]]}.

\subsection{Using MACROFORT}
%%%%%%%%%%%%%%%%%%%%%%%%%%%%

You have to load MACROFORT before using it by issuing the commands:
\begin{verbatim}
read ``.libname.`/macrofort.m`:
\end{verbatim}
if MACROFORT is installed into MAPLE library.

Then, two new functions are available:
\begin{description}
  \item[init\_genfor] must be used without argument before a new FORTRAN code
generation. It gives their default values to the global variables used
by MACROFORT and it initializes the various counters used internally by
MACROFORT (for label generation for example).
  \item[genfor] is the function which really generates the FORTRAN code. Its
argument is the MAPLE list describing the FORTRAN program.
\end{description}

In summary, to generate FORTRAN code in a file, you have to executes
the following MAPLE commands:
\begin{verbatim}
flist:=[...]; # LIST DESCRIBING THE FORTRAN PROGRAM
writeto(`foo.fortran`);
init_genfor(); # INITIALIZATION OF MACROFORT
interface(quiet=true); # NOT TO HAVE MESSAGES FROM MAPLE
genfor(flist);
interface(quiet=false);
writeto(terminal);
\end{verbatim}

%%%%%%%%%%%%%%%%%%%%%%%%%%%%%%%%%%%%%%%%%%%%%%%%%%%%%%%%%%%%%%%%%%%%%%%%%%%%%%
\section{Example 1: Generalized Newton method}
%%%%%%%%%%%%%%%%%%%%%%%%%%%%%%%%%%%%%%%%%%%%%%%%%%%%%%%%%%%%%%%%%%%%%%%%%%%%%%
\label{e1}

\subsection{Newton algorithm}
%%%%%%%%%%%%%%%%%%%%%%%%%%%%%

We want to solve a non-linear system with $n$ equations and $n$ unknowns:
\[ \left\{
    \begin{array}{lcl}
      f_1(x_1,\ldots ,x_n) & \four!    =   \four! & 0 \\
      \vdots               & \four! \vdots \four! & \\
      f_n(x_1,\ldots ,x_n) & \four!    =   \four! & 0 \\
    \end{array}
  \right. \]
with $x_1,\ldots ,x_n$ belonging to $\R$.
We use the matrix notation defining the vectors $X$ and $F(X)$, and 
the Jacobian matrix of the system $F'(X)$---a $n$ by $n$ matrix---as :
\[
  X\is\left(\begin{array}{c} x_1 \\ \vdots \\ x_n \end{array}\right) \qquad
  F(X)\is\left(
    \begin{array}{c} f_1(x_1,\ldots,x_n) \\ \vdots \\f_n(x_1,\ldots,x_n)
    \end{array}\right) \qquad
  F'(X)\is{\left(\frac{\partial f_i}{\partial x_j}\right)}_{n,n}
\]

The well-known algorithm of Newton method 
is---in Pascal like syntax--- :\smallskip\par
$X:=X0$ \{ where X0 is an arbitrary vector \} \par
{\bf while} $NORM(F(X)) > EPS$ {\bf do} \{ where $NORM$ computes the norm
  of $F(X)$ \} \par
\quad $<$ find $Y$ solution of the linear system $F'(X).Y =-F(X)$ $>$ \par
\quad $X:=Y+X$ \par
{\bf end.}

The problem when encoding this algorithm in FORTRAN code is to compute
the jacobian.

\subsection{MACROFORT program}
%%%%%%%%%%%%%%%%%%%%%%%%%%%%%%

We give below a MAPLE function which generates a general FORTRAN program
for solving a non-linear system by using MACROFORT.

{\tt resol} is a FORTRAN subroutine solving a linear system of equations.

\begin{verbatim}
# GENERALIZED NEWTON ALGORITHM
#  F[I] IS THE TABLE OF THE FUNCTIONS WITH THE VARIABLES X[J]
#  X IS THE GENERIC NAME OF THE VARIABLES X[J]
#  N IS THE NUMBER OF FUNCTIONS AND VARIABLES 

gen_newton := proc(ff,xx,n)
  local f,fo,ii,jac,jj,linit,lwhile,m,pg,x,zj;
# main program
# Jacobian matrix
  jac := array(1..n,1..n);
  for ii from 1 to n do
    for jj from 1 to n do
      jac[ii,jj] := diff(ff[ii],xx[jj])
  od od;
# pg: MAPLE list describing the main program
  pg:=[[declaref,real,[f[n],zj[n,n],x[n]]],
       [writem,output,[`'eps = '`],[]],
       [readm,input,[`e14.7`],[eps]]];
# while instruction
  linit:=[[dom,i,1,n,
          [[writem,output,[`'x(',i3,') = '`],[i]],
           [readm,input,[`e14.7`],[x[i]]]]]];
  for ii from 1 to n do pushe([equalf,f[ii],ff[ii]],'linit') od;
  lwhile:=[[matrixm,zj,jac],
              [callf,resol,[zj,f,n]],
              [dom,i,1,n,[equalf,x[i],-f[i]+x[i]]]];
  for ii from 1 to n do pushe([equalf,f[ii],ff[ii]],'lwhile') od;
  pg:=[op(pg),
             [[whilem,znorm(f,n) >= eps,linit,lwhile,1000],
              [writem,output,[`(2x,e14.7)`],[x]]]];
 pg:=[programm,newton,pg];

# fo: MAPLE list describing the subroutine computing the quadratic norm of f
  fo:=[[declaref,real,[f[m]]],
      [equalf,znorm,0.],
      [dom,i,1,n,[equalf,znorm,znorm+f[i]**2]],
      [equalf,znorm,sqrt(znorm)]];
  fo:=[functionm,real,znorm,[f,m],fo];

# FORTRAN code generation
  writeto(`newton.f`);
  init_genfor();
  interface(quiet=true);
  genfor(pg);
  genfor(fo);
  interface(quiet=false);
  writeto(terminal);
end:
\end{verbatim}

\subsection{Application to a steel rolling problem}
%%%%%%%%%%%%%%%%%%%%%%%%%%%%%%%%%%%%%%%%%%%%%%%%%%%

The system we want to solve is described by the equations:
\begin{eqnarray*}
 f_1(F,h_2,\Phi)& = & h_2-S-\frac{F+a_2(1-e^{a_3 F})}{a_1} \\
 f_2(F,h_2,\Phi)& = & F-l k R 
   \left(\frac{1}{2} \pi \sqrt{\frac{h_2}{R}}\arctan\sqrt{r}
     -\frac{\pi\xi}{4}-\ln\left(\frac{h_N}{h_2}\right)
     +\frac{1}{2}\ln\left(\frac{h_1}{h_2}\right)\right) \\
 & & +\frac{R \xi T_1}{h_2} \\
 f_3(F,h_2,\Phi)& = & \arctan\left(\Phi\sqrt{\frac{R}{h_2}}\right) \\
 & & -\frac{1}{2}\sqrt{\frac{h_2}R}
       \left(\frac{\pi}4 \ln\left(\frac{h_2}{h_1}\right)
         +\sqrt{\frac{R}{h_2}} \arctan\sqrt{r}-\frac{T_1}{k l h_1}
         +\frac{T_2}{k l h_2}\right)
\end{eqnarray*}
with
\[
 r=\frac{h_1-h_2}{h_2} \qquad
 \xi=\sqrt{\frac{h_1-h_2}R} \qquad
 h_N=h_2+R\Phi^2
\]
and with the following values of the parameters:
$a_1=610$, $a_2=648$, $a_3=-.00247$, $l=1250$, $k=.0014$, $R=360$,
$T_1=12$, $T_2=35$, $h_1=24$ and $S=12$.

This set of equations is the simplified description of the behavior 
of a steel strip
in a stand of a hot strip rolling mill.

If you want to solve it by using pure MAPLE, you cannot use {\tt solve} to
obtain the analytic solution because the equations are too complicated.
Then, you can use {\tt fsolve} to solve numerically the problem. But
{\tt fsolve} gives a bad solution which has no physical meaning. So you
are obliged to use a FORTRAN program.

\medskip
The MAPLE program which generates the corresponding FORTRAN code is:
\begin{verbatim}
# STEEL ROLLING
#   solution by Newton method

# numerical values of parameters
data:=[a1=610,a2=648,a3=-.00247,l=1250,k=1.4*10^(-2),gr=360,t1=12,
  t2=35,h1=24,s=12]:

# equations
exp1:=h2-s-(f+a2*(1-exp(a3*f)))/a1:
exp2:=f-l*k*gr*(Pi*sqrt(h2/gr)*arctan(sqrt(r))/2-Pi*csi/4-log(hn/h2)+
              log(h1/h2)/2)
      +gr*csi*t1/h2:
exp3:=arctan(phi*sqrt(gr/h2))-sqrt(h2/gr)*(Pi*log(h2/h1)/4+sqrt(gr/h2)*
     arctan(sqrt(r))-t1/k/l/h1+t2/k/l/h2)/2:
r:=(h1-h2)/h2:
csi:=sqrt((h1-h2)/gr):
hn:=h2+gr*phi^2:

# table of equations with good variables
f[1] := subs([op(data),f=x[1],h2=x[2],phi=x[3]],exp1):
f[2] := subs([op(data),f=x[1],h2=x[2],phi=x[3]],exp2): 
f[3] := subs([op(data),f=x[1],h2=x[2],phi=x[3]],exp3):

# loading MACROFORT
read ``.libname.`/macrofort.m`:

# reading and executing the MAPLE procedure generating the FORTRAN code
read gen_newton:
gen_newton(f,x,3);
\end{verbatim}

\subsection{Generated FORTRAN program}
%%%%%%%%%%%%%%%%%%%%%%%%%%%%%%%%%%%%%%

We give below the FORTRAN program generated by MACROFORT.

\begin{verbatim}
c      
c     MAIN PROGRAM newton
c      
      program newton
        real f(3),zj(3,3),x(3)
        write(6,2000) 
 2000   format('eps = ')
        read(5,2001) eps
 2001   format(e14.7)
c          
c         WHILE  (eps<=znorm(f,3)) DO <WHILE_LIST> (1)
c          
c         WHILE LOOP INITIALIZATION
              maxwhile1 = 1000
              nwhile1 = 0
c             
            do 1000, i=1,3
              write(6,2002) i
 2002         format('x(',i3,') = ')
              read(5,2003) x(i)
 2003         format(e14.7)
 1000       continue
c           
            f(1) = x(2)-3984.D0/305.D0-x(1)/610+324.D0/305.D0*dexp(-0.24
     +7D-2*x(1))
            f(2) = x(1)-0.8750000001D1*0.3141592653589793D1*dsqrt(x(2))*
     +dsqrt(360.D0)*datan(dsqrt(24-x(2))/dsqrt(x(2)))+0.4375D1*0.3141592
     +653589793D1*dsqrt(24-x(2))*dsqrt(360.D0)+0.63D4*dlog((x(2)+360*x(3
     +)**2)/x(2))-0.315D4*dlog(24/x(2))+12*dsqrt(360.D0)*dsqrt(24-x(2))/
     +x(2)
            f(3) = datan(x(3)*dsqrt(360.D0)/dsqrt(x(2)))-dsqrt(x(2))*dsq
     +rt(360.D0)*(0.3141592653589793D1*dlog(x(2)/24)/4+dsqrt(360.D0)/dsq
     +rt(x(2))*datan(dsqrt(24-x(2))/dsqrt(x(2)))-0.2857142857D-1+0.2D1/x
     +(2))/720
c          
c         WHILE LOOP BEGINNING
 1001     continue
c          
c         WHILE LOOP TERMINATION TESTS
          if (eps.le.znorm(f,3.D0)) then
            if (nwhile1.le.maxwhile1) then
c              
c             NEW LOOP ITERATION
              nwhile1 = nwhile1+1
c              
c             <WHILE_LIST>
              zj(1,1) = -1.D0/610.D0-0.2623868853D-2*dexp(-0.247D-2*x(1)
     +)
              zj(3,1) = 0
              zj(2,1) = 1
              zj(3,3) = dsqrt(360.D0)/dsqrt(x(2))/(1+360*x(3)**2/x(2))
              zj(2,2) = -0.4375000001D1*0.3141592653589793D1/dsqrt(x(2))
     +*dsqrt(360.D0)*datan(dsqrt(24-x(2))/dsqrt(x(2)))-0.8750000001D1*0.
     +3141592653589793D1*dsqrt(x(2))*dsqrt(360.D0)*(-1/(dsqrt(24-x(2)))/
     +dsqrt(x(2))/2-dsqrt(24-x(2))/dsqrt(x(2)**3)/2)/(1+(24-x(2))/x(2))-
     +0.21875D1*0.3141592653589793D1/dsqrt(24-x(2))*dsqrt(360.D0)+0.63D4
     +*(1/x(2)-(x(2)+360*x(3)**2)/x(2)**2)/(x(2)+360*x(3)**2)*x(2)+0.315
     +D4/x(2)-6*dsqrt(360.D0)/dsqrt(24-x(2))/x(2)-12*dsqrt(360.D0)*dsqrt
     +(24-x(2))/x(2)**2
              zj(2,3) = 0.4536D7*x(3)/(x(2)+360*x(3)**2)
              zj(1,2) = 1
              zj(3,2) = -x(3)*dsqrt(360.D0)/dsqrt(x(2)**3)/(1+360*x(3)**
     +2/x(2))/2-1/dsqrt(x(2))*dsqrt(360.D0)*(0.3141592653589793D1*dlog(x
     +(2)/24)/4+dsqrt(360.D0)/dsqrt(x(2))*datan(dsqrt(24-x(2))/dsqrt(x(2
     +)))-0.2857142857D-1+0.2D1/x(2))/1440-dsqrt(x(2))*dsqrt(360.D0)*(0.
     +3141592653589793D1/x(2)/4-dsqrt(360.D0)/dsqrt(x(2)**3)*datan(dsqrt
     +(24-x(2))/dsqrt(x(2)))/2+dsqrt(360.D0)/dsqrt(x(2))*(-1/(dsqrt(24-x
     +(2)))/dsqrt(x(2))/2-dsqrt(24-x(2))/dsqrt(x(2)**3)/2)/(1+(24-x(2))/
     +x(2))-0.2D1/x(2)**2)/720
              zj(1,3) = 0
              call resol(zj,f,3)
c               
              do 1002, i=1,3
                x(i) = -f(i)+x(i)
 1002         continue
c             
              f(1) = x(2)-3984.D0/305.D0-x(1)/610+324.D0/305.D0*dexp(-0.
     +247D-2*x(1))
              f(2) = x(1)-0.8750000001D1*0.3141592653589793D1*dsqrt(x(2)
     +)*dsqrt(360.D0)*datan(dsqrt(24-x(2))/dsqrt(x(2)))+0.4375D1*0.31415
     +92653589793D1*dsqrt(24-x(2))*dsqrt(360.D0)+0.63D4*dlog((x(2)+360*x
     +(3)**2)/x(2))-0.315D4*dlog(24/x(2))+12*dsqrt(360.D0)*dsqrt(24-x(2)
     +)/x(2)
              f(3) = datan(x(3)*dsqrt(360.D0)/dsqrt(x(2)))-dsqrt(x(2))*d
     +sqrt(360.D0)*(0.3141592653589793D1*dlog(x(2)/24)/4+dsqrt(360.D0)/d
     +sqrt(x(2))*datan(dsqrt(24-x(2))/dsqrt(x(2)))-0.2857142857D-1+0.2D1
     +/x(2))/720
            goto 1001
            else
c              
c             WHILE LOOP TERMINATION :
c             BYPASSING THE MAXIMUM ITERATION NUMBER
              write(6,2004) 
 2004         format(' maxwhile1 ')
            endif
c          
c         NORMAL WHILE LOOP TERMINATION
          endif
c         WHILE LOOP END (1)
          write(6,2005) x
 2005     format((2x,e14.7))
      end
c      
c     FUNCTION znorm
c      
      real function znorm(f,m)
        real f(m)
        znorm = 0
c         
        do 1003, i=1,3
          znorm = znorm+f(i)**2
 1003   continue
c       
        znorm = dsqrt(znorm)
      end
\end{verbatim}

\newpage

%%%%%%%%%%%%%%%%%%%%%%%%%%%%%%%%%%%%%%%%%%%%%%%%%%%%%%%%%%%%%%%%%%%%%%%%%%%%%%
\section{Example 2: A recursive function on a tree}
%%%%%%%%%%%%%%%%%%%%%%%%%%%%%%%%%%%%%%%%%%%%%%%%%%%%%%%%%%%%%%%%%%%%%%%%%%%%%%
\label{e2}

It is possible to write FORTRAN program solving recursive problems.

But it can be very tedious and very complicated to do it when the
recursion is complicated.

When the recursion is not so complicated, we can have other problems.

For example, we consider a binary tree with nodes labeled by a couple
of integers $(i,j)$. $(1,1)$ is the root of the tree, $(2,1)$ and $(2,2)$
are the children nodes of the root and recursively $(i+1,2j-1)$ and
$(i+1,2j)$ are the children nodes of node $(i,j)$. Node $(i,j)$ is
at level $i$.

Assume we have the following sequence:
\[ \begin{array}{l}
f_{1,1}\ \mbox{given}\\
f_{i,j}=\left\{ \begin{array}{ll}
  g(f_{i-1,\frac{j}{2}}) & \mbox{if $j$ is even}\\
  g(f_{i-1,\frac{j+1}{2}}) & \mbox{if $j$ is odd} \end{array} \right.
\end{array} \]
where $g$ is a given function.

We want to compute the values of the sequence up to a given level $N$, \ie
the $2^{N-1}$ values\\ $f(N,1) \ldots f(N,2^{N-1})$.

To write the corresponding FORTRAN code, you only have to write
two loops:
\begin{verbatim}
      real f(n,m)
      do 1,i=1,n
        do 2,j=1,2**(n-1)-1,2
            f(i,j)=g(f(i-1,(j+1)/2))
 2      continue
        do 3,j=2,2**(n-1),2
            f(i,j)=g(f(i-1,j/2))
 3      continue
 1    continue
\end{verbatim}
but the dimension {\tt m} of the array {\tt f} is $2^{N-1}$ ! 
We have to keep the
storage for $N \times 2^{N-1}$ real values instead of $2^N-1$ real 
values (5 times 
more storage for $N$ equal to 10).

A way to solve this problem is to have an array for each level, \ie
to have in the FORTRAN program arrays {\tt f1(1)}, {\tt f2(2)},
{\tt f3(4)} and so on. But now it is very tricky to write the
FORTRAN program.

It is easy to generate this program with MACROFORT.
We suppose that a FORTRAN function {\tt g} has already been defined.
The MAPLE function which generates the FORTRAN program is:
\begin{verbatim}
gen_func := proc(n)
  local i,j,pg;
  pg:=[];
# declaration of the arrays
  for i from 1 to n do
    pushe([declaref,real,[f.i[2^(i-1)]]],'pg');
  od;
# loops for each array
  for i from 2 to n do
    pushe([dom,j,1,2**(i-1)-1,2,
            [equalf,f.i[j],g(f.(i-1)[`(j+1)`/2])]],'pg');
    pushe([dom,j,2,2**(i-1),2,
            [equalf,f.i[j],g(f.(i-1)[j/2])]],'pg');
  od;
  pg:=[programm,func,pg];
  writeto(`func.f`);
  init_genfor();
  interface(quiet=true);
  genfor(pg);
  interface(quiet=false);
  writeto(terminal);
end:
\end{verbatim}

The generated FORTRAN program for $N$ equal to 4 is:
\begin{verbatim}
c      
c     MAIN PROGRAM func
c      
      program func
        real f1(1)
        real f2(2)
        real f3(4)
        real f4(8)
        real f5(16)
c         
        do 1000, j=1,1,2
          f2(j) = g(f1((j+1)/2))
 1000   continue
c       
c         
        do 1001, j=2,2,2
          f2(j) = g(f1(j/2))
 1001   continue
c       
c         
        do 1002, j=1,3,2
          f3(j) = g(f2((j+1)/2))
 1002   continue
c       
c         
        do 1003, j=2,4,2
          f3(j) = g(f2(j/2))
 1003   continue
c       
c         
        do 1004, j=1,7,2
          f4(j) = g(f3((j+1)/2))
 1004   continue
c       
c         
        do 1005, j=2,8,2
          f4(j) = g(f3(j/2))
 1005   continue
c       
c         
        do 1006, j=1,15,2
          f5(j) = g(f4((j+1)/2))
 1006   continue
c       
c         
        do 1007, j=2,16,2
          f5(j) = g(f4(j/2))
 1007   continue
c       
      end
\end{verbatim}

Together with the previous problem of storage, we can have a much more
complicated recursion. For example the $g$ function can be dependent of
the node of the tree where we compute the value of the sequence.
Such a problem occurs in multiscale statistical signal processing when using
wavelet transforms and we want to compute modeling filters by Schur-Levinson
recursions (see \cite{wave}). Investigations to solve this problem are
undertaken.

%%%%%%%%%%%%%%%%%%%%%%%%%%%%%%%%%%%%%%%%%%%%%%%%%%%%%%%%%%%%%%%%%%%%%%%%%%%%%%
\section{Conclusion}
%%%%%%%%%%%%%%%%%%%%%%%%%%%%%%%%%%%%%%%%%%%%%%%%%%%%%%%%%%%%%%%%%%%%%%%%%%%%%%
\label{conclusion}

MACROFORT has been described. We have shown that it is possible to 
make FORTRAN code within MAPLE using it as a preprocessor for numerical
analysis. Moreover MACROFORT has shown its capability for:
\begin{itemize}
  \item solving numerical problems that MAPLE cannot manage
  \item doing mathematical computations before numerical computation
  \item solving storage problems in FORTRAN code for classes of recursive
algorithms
  \item solving numerically recursive algorithms.
\end{itemize}

\begin{thebibliography}{99}
  \bibitem{macsyma} Chancelier J.P., Gomez C. and Quadrat J.P.,
    MACROFORT : a FORTRAN Code Generator in MACSYMA, MACSYMA Newsletter,
    1987.
  \bibitem{wave} Basseville M. and Benveniste A., Multiscale Statistical
    Signal Processing, INRIA report 970, 1989.
\end{thebibliography}

\end{document}
